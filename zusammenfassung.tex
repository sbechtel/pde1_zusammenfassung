\documentclass[11pt,a4paper]{scrartcl}

\usepackage[utf8]{inputenc}
\usepackage[T1]{fontenc}
\usepackage[ngerman]{babel}
\usepackage{amsmath,amsthm,amssymb,dsfont}
\usepackage{mathtools}
\usepackage[paper=a4paper,left=25mm,right=25mm,top=25mm,bottom=25mm]{geometry}
\usepackage{float}
\usepackage{hyperref}
\usepackage{enumerate}
\usepackage{url}
\usepackage{tikz}
\usepackage{esint}
\usepackage{csquotes}
\usepackage{textcomp}

\usepackage{setspace}

\parindent 0pt
\linespread{1.5}

% Makros

\newcommand{\N}{\mathbb{N}} % natuerliche Zahlen
\newcommand{\Z}{\mathbb{Z}} % ganze Zahlen
\newcommand{\Q}{\mathbb{Q}} % rationale Zahlen
\newcommand{\R}{\mathbb{R}} % reelle Zahlen
\newcommand{\K}{\mathbb{K}} % Körper
\newcommand{\C}{\mathbb{C}} % komplexe Zahlen
\newcommand{\D}{\mathcal{D}}
\newcommand{\E}{\mathcal{E}}
\newcommand{\Sc}{\mathcal{S}}
\newcommand{\F}{\mathcal{F}}
\newcommand{\norm}[1]{\|#1\|}
\newcommand{\laplace}{\triangle}
\newcommand{\circum}{\text{\textasciicircum}}
\newcommand{\dd}{\frac{\mathrm{d}}{\mathrm{d}x}}

% Umgebungen für Definitionen, Sätze, usw.

\theoremstyle{plain}
\newtheorem{thm}{Satz}[section]
\newtheorem*{lem}{Lemma}
\newtheorem{cor}[thm]{Korollar}
\newtheorem{prop}[thm]{Proposition}
\newtheorem*{ex}{Beispiel}
\newtheorem*{ntion}{Notation}

\theoremstyle{definition}
\newtheorem{defn}[thm]{Definition}

\theoremstyle{remark}
\newtheorem*{bem}{Bemerkung}
\newtheorem{bemnumber}[thm]{Bemerkung}

\def\Satzrefname{Satz}

\DeclareMathOperator{\supp}{supp}
\DeclareMathOperator{\esssupp}{ess supp}
\DeclareMathOperator{\id}{id}
\DeclareMathOperator{\loc}{loc}
\DeclareMathOperator{\pv}{pv}
\DeclareMathOperator{\grad}{grad}
\DeclareMathOperator{\e}{e}

\begin{document}

\title{Zusammenfassung PDE I}
\author{Sebastian Bechtel}
\maketitle

\section{lineare Grundgleichungen}

\subsection{Transportgleichung}

Betrachte $u_t + a\cdot u_x = 0$. \emph{Methode der Charakteristiken}: Aus der DGL folgt, dass eine Lösung konstant längs $s\mapsto (s,as+c)$ ist. Bestimme $u(x,y)$ aus den Anfangswerten durch eine Kurve obiger Form, die durch $(x,y)$ und den Definitionsbereich der Anfangswerte geht.

\subsection{Laplace-Gleichung}

Betrachte $\laplace u = 0$. Lösungen heißen \emph{harmonische Funktionen}. Nutze Rotationsinvarianz von $\laplace$, um ODE in $r$ zu erhalten. Nichttriviale Lösung heißt \emph{Fundamentallösung des Laplace}. Für $n=2$: $C \log |x|$, für $n \geq 3$: $C |x|^{2-n}$ (jeweils für $x\in \R \setminus \{0\}$, Ansatz benötigt Differenzierbarkeit von $r=|x|$).

\subsubsection{Anschauung: harmonische Funktionen auf $\R$}

Sei $f: \R \to \R$ harmonisch, also $f''=0$ auf $\R$. Dann ist $f$ konvex und konkav. Betrachte $f$ eingeschränkt auf $[a,b]$. Dann muss der Graph von $f$ oberhalb und unterhalb der Verbindungslinie zwischen $f(a)$ und $f(b)$ liegen, also ist $f$ affin linear. Somit ist klar, dass $f$ entweder konstant ist, oder die Extrema am Rand angenommen werden. Außerdem rechnet man leicht $f(x)=\frac{1}{2h} \int_{x-h}^{x+h} f(y) \, \mathrm{d}y$ nach, also gilt die Mittelwerteigenschaft.

\subsubsection{Poisson-Gleichung (auf dem Ganzraum)}

Betrachte $-\laplace u = f$, wobei $f\in C_c^2(\R^n)$. Eine Lösung ist gegeben durch $u(x)\coloneqq (\varphi * f)(x)$. Trick: Schneide Singularität der Fundamentallösung aus dem Faltungsintegral raus.

\subsubsection{Mittelwerteigenschaft}

Eine Funktion  $u\in C^2(\Omega)$ ist harmonisch gdw. Mittelwerteigenschaft $$u(x) = \fint_{B_r(x)} u(y) \, \mathrm{dy}$$ gilt.

\subsubsection{Maximumsprinzip}

Ist $\Omega$ beschränktes Gebiet, $u\in C^2(\Omega) \cap C(\overline{\Omega})$ harmonisch, dann gelten die folgenden Aussagen:

\begin{enumerate}
    \item \emph{Schwaches Maximumsprinzip}: $\max_{\overline{\Omega}} u = \max_{\partial \Omega} u$.
    \item \emph{Starkes Maximumsprinzip}: Ist $x_0 \in \Omega$ mit $u(x_0) = \max_{\overline{\Omega}} u$, so ist $u$ konstant.
\end{enumerate}

Beweis: Starkes Maximumsprinzip folgt aus der MWE: $|u(x_0)| \leq \fint_{B_r(x)} |u(y)| \, \mathrm{d}y \leq \fint_{B_r(x)} |u(x_0)| \, \mathrm{d}y = |u(x_0)|$, also muss $u$ konstant auf dem Ball sein, also durch Ausschöpfung auf dem Gebiet. Schwaches Maximumsprinzip folgt unmittelbar aus dem starken.

\subsubsection{Eindeutigkeit der Lösung des DP}

Betrachte $-\laplace u = f$ in $\Omega$, $u=g$ auf $\partial \Omega$. Sind $u_1,u_2$ Lösungen, so betrachte $w_1=u_1-u_2$, $w_2=u_2-u_1$. Da $w_i$ die Laplace-Gleichung löst, folgt mit dem Maximumsprinzip: $\max_{\overline{\Omega}} u_1-u_2 = \max_{\partial \Omega} w_1 = 0$, also $u_2\leq u_1$ und analog $u_1\leq u_2$.

\subsubsection{Glattheit harmonischer Funktionen}

Ist $u\in C(\Omega)$ mit Mittelwerteigenschaft, so ist $u\in C^\infty(\Omega)$, somit $h$ harmonisch. Beweis: Geschickte Nutzung von Mollifiern. Alternativer Beweis: Jede harmonische Funktion ist Realteil einer holomorphen Funktion. Holomorphe Funktionen sind komplex unendlich oft differenzierbar, also auch reell unendlich oft differenzierbar, also ist der Realteil eine glatte Funktion.

\subsubsection{Green'sche Funktionen}

Betrachte Poisson-Problem auf beschränktem Gebiet $\Omega$. Leite Lösungsformel her; Diese enthält $\frac{\partial u}{\partial \nu}$ (unbekannt). Löse $\laplace \Phi^x = 0$ in $\Omega$, $\Phi^x = \Phi(y-x)$ auf $\partial \Omega$ für alle $x\in\Omega$. Durch $G(x,y)\coloneqq \Phi(y-x)-\Phi^x(y)$ wird \emph{Green-Funktion} definiert. Diese liefert Lösungsformel (für das Poisson-Problem auf diesem Gebiet), die nur bekannte Größen enthält. Problem: Bestimme zu gegebenem Gebiet $\Omega$ die Lösungen $\Phi^x$ der obigen DGL. Idee: Entferne Singularität durch \enquote{Reflexion} aus dem Gebiet.

\subsubsection{Eindeutigkeit des DP mit Energiemethode}

Zeige für $w\coloneqq u_1-u_2$, dass $|\nabla w|=0$ gilt (Green'sche Formeln). Da $w$ am Rand $0$, im inneren konstant und stetig, folgt $u_1=u_2$.

\subsection{Wärmeleitungsgleichung}

Betrachte $u_t-\triangle u = 0$ auf $(0,\infty)\times \R^n$, $u(0,x)=u_0(x)$ auf $\R^n$. Fourier-Trafo in $x$ liefert ODE für $\hat u$, jene ist explizit lösbar, Rücktrafo. liefert Fundamentallösung $G(t,x)\coloneqq (4\pi t)^{-n/2} \, e^{-\frac{|x|^2}{4t}}$ für $t>0$, sonst $0$, genannt \emph{Gauß-Kern}. Die Faltung $u(t,x)\coloneqq (G_t*u_0)(x)$ liefert Lösung. Der Gauß-Kern glättet, es gilt $u\in C^\infty$.

\subsubsection{Mittelwerteigenschaft}

Es gilt eine Mittelwerteigenschaft für Heatballs im parabolischen Zylinder. Ein Heatball ist gegeben durch $(x,t) + \{(y,s): \Phi(y,s) > \lambda\}$, $(x,t)$ ist der "Mittelpunkt" des Heatballs, sieht ungefähr aus wie ein Ellipsoid, dessen oberster Punkt der Mittelpunkt ist. Hängt wegen der Heatball-Definition nur von vergangenen Zeiten ab.

\subsubsection{Maximumsprinzip}

Wie bei $\triangle$, jedoch nun mit parabolischem Zylinder und parabolischem Rand.

\subsubsection{Eindeutigkeit der Lösung der Wärmeleitungsgleichung}

Wie bei der Poisson-Gleichung mittels Maximumsprinzip.

\subsection{Wellengleichung}

Betrachte $u_{tt}-\triangle u = 0$ in $\R\times \R^n$. 

Fall $n=1$: Reduziere auf Transportgleichung ($\mathrm{d}t^2-\mathrm{d}x^2=(\mathrm{d}t+\mathrm{d}x)(\mathrm{d}t-\mathrm{d}x)$, substituiere $v=u_t-u_x$ und löse zweimal Transportgleichung). Lösung ist so glatt wie Anfangswert. 

Fall $n$ ungerade: Reduziere auf parabolische Gleichung. Rechnen und Laplace-Trafo liefert Lösungsformel, in $n=3$ \emph{Kirschhoff'sche Formel} genannt. Lösung ist $C^2$, Regularitätsforderung steigt in der Dimension, die Gleichung glättet also nicht! Lösung hängt vom Anfangswert nur auf Sphären ab. 

Fall $n$ gerade: Ist $u(t,x_1,\dots,x_n)$ eine Lösung in Dimension $n$, so ist $\bar u(t,x_1,\dots,x_n,x_{n+1})=u(t,x_1,\dots,x_n)$ eine Lösung in Dimension $n+1$, also ungerade, dort haben wir aber eine Lösungsformel. Dann rumrechnen (Transformation von Ball auf Halbraum). Lösung hängt vom Anfangswert auf Bällen ab. Die Lösungen unterscheiden sich also fundamental in geraden und ungeraden Raumdimensionen!

\subsubsection{Eindeutigkeit der Lösung der Wellengleichung}

Energiemethode.

\subsubsection{Kein Maximumsprinzip für die Wellengleichung}

Betrachte $u(t,x)\coloneqq \sin(t+x)$. Dann ist $u$ Lösung der Wellengleichung auf $\R^2$. Betrachte nun $u$ auf $\Omega \coloneqq \{ (t,x): t+x \leq 2\pi \}$. Dann gilt $u=0$ auf $\partial\Omega$ und $u$ nimmt sein Maximum $1$ und sein Minimum $-1$ im Inneren auf $\{ (t,x): t+x=\pi/2 \}$ bzw. $\{ (t,x): t+x=3/2\pi \}$ an.

\section{schwache Lösungstheorie}

\subsection{Sobolevräume}

Es bezeichne $W^{1,p}(\Omega)$ den Sobolevraum mit einer schwachen Ableitung in $L^p(\Omega)$. Im Fall $p=2$ schreiben wir $H^1(\Omega)$. Schwache Ableitungen sind nach Fundamentallemma eindeutig bestimmt. Wir definieren in $W^{1,p}(\Omega)$ den Teilraum $W^{1,p}_0(\Omega) \coloneqq \overline{C_c^\infty(\Omega)}^{\|\cdot\|_{W^{1,p}(\Omega)}}$. Es sind $H^1(\Omega)$ und $H^1_0(\Omega)$ Hilberträume.

Für $\Omega=(a,b)$ hat $u\in W^{1,p}(a,b)$ einen stetigen Representanten: Fixiere $x_0\in (a,b)$ und setze $\tilde u(x)=\int_{x_0}^x u'$. Dann $\int (\tilde u-u)\varphi'=\int -u'\varphi-u\varphi'=\int (u'-u')\varphi=0$, also $\tilde u-u=C$ f.ü., also $\tilde u-C$ stetiger Repräsentant (sogar gleichmäßig stetig, also auch in $C([a,b])$!). Insbesondere gilt also auch der Hauptsatz!

Zusätzlich ist die Einbettung $W^{1,p}(a,b) \hookrightarrow C(a,b)$ stetig: $|u(x)| \leq |C| + \int_{x_0}^x |u'| \leq |C| + \int |u'| \leq |C| +\|u\|_{W^{1,p}(a,b)}$. Für $u\in C^1(a,b)$ können wir nach Mittelwertsatz der Integralrechnung $|C| \leq 1/(b-a) \|u\|_{L^1(a,b)}$ realisieren.

Aus der gleichmäßigen Stetigkeit der Repräsentanten folgt die Existenz eines stetigen Randoperators $W^{1,p}(a,b)\to L^p(\partial (a,b))=L^p(\{a,b\})$: $\|u\|_{L^p(\{a,b\})} = |u(a)|+|u(b)| \leq 2 \|u\|_{C(a,b)} \leq C \|u\|_{W^{1,p}(a,b)}$.

Diese Resultate lassen sich verallgemeinern in höhere Raumdimensionen und auf Lipschitz-Gebiete, dabei sind ersteres Sobolev-Einbettungssätze und letzteres die Existenz eines Trace-Operators. Die Strategie ist dann, auf geeigneten dichten Teilmengen solche Abschätzungen zu zeigen. In $\R^d$ ist $C_c^\infty(\R^d)$ eine solche dichte Teilmenge (nutze Mollifier und Cut-Off-Funktion). Außerdem gibt es für Lipschitz-Gebiete einen Fortsetzungsoperator $W^{1,p}(\Omega) \to W^{1,p}(\R^d)$, somit ist $C^\infty(\Omega)$ (Achtung: kein kompakter Träger mehr!) dicht in $W^{1,p}(\Omega)$.

Ist $\alpha\in (0,1)$ und $k,r \in \N$, $p\geq 1$ sodass $(k-r-\alpha)/n=1/p$, dann gibt es eine stetige Einbettung $W^{k,p}(\Omega) \hookrightarrow C^{r,\alpha}(\Omega)$.

Außerdem kann man Regularität investieren, um Integrierbarkeit zu erhalten: ist $k>l$ und $1\leq p \leq q < \infty$ mit $k-l < n/p$ und $\frac{1}{q}=\frac{1}{p}-\frac{k-l}{n}$, dann $W^{k,p}(\Omega) \hookrightarrow W^{l,q}(\Omega)$.

\subsection{schwache Lösung des Dirichlet-Problem}

Betrachte wieder $-\laplace u = f$ in $\Omega$, $u=0$ auf $\partial\Omega$. Eine Funktion $u\in H^1_0(\Omega)$ mit $\int \nabla u \nabla v = \int fv$ für alle $v\in H^1_0(\Omega)$ heißt \emph{schwache Lösung des (DP)}. 

Hat $\Omega$ glatten Rand, so gilt $u\in H^1_0(\Omega)$ gdw. $u=0$ auf $\partial\Omega$ für $u\in H^1(\Omega)$ (im Spursinne).

Klassische Lösung ist schwache Lösung: Aus der Randwertbedingung folgt $u\in H^1_0(\Omega)$. Bedingung der schwachen Lösung auf $C_c^\infty(\Omega)$ nachrechnen, dann Dichtheitsargument.

Existenz und Eindeutigkeit der schwachen Lösung: Es gilt für $u\in H^1_0(\Omega)$ die Poincare-Ungleichung $\norm{u}_{L^2} \leq C \norm{\nabla u}_{L^2}$, wobei $C$ nur von $\Omega$ abhängt. Die Formen $a(u,v)\coloneqq \int \nabla u \nabla v$ sowie $f(v)\coloneqq \int fv$ sind beschränkt. Außerdem zeigt man durch $|\nabla u|^2 = 1/2|\nabla u|^2 + 1/2|\nabla u|^2$ und Poincare, dass $a$ koerziv ist, also liefert Lax-Malgram die Existenz einer eindeutigen Schwachen Lösung.

Regularität: Für $f\in H^m(\Omega)$ gilt, dass die schwache Lösung $u$ des zugehörigen (DP) die Regularität $H^{m+2}(\Omega)$ besitzt.

Rückkehr zur klassischen Lösung: Es gilt ein Lemma von Sobolev: ist $m > n/2 + k$ und $u\in H^m(\Omega)$, so gibt es $g\in C^k(\Omega)$ mit $g=u$ f.ü. Beweisidee: Fouriertrafo und $u\in H^m(\Omega)$ gdw. $(1+|\xi|^2)^{m/2} |\hat f(\xi)| \in L^2(\R^n)$. Das die schwache Lösung gleich der starken Lösung ist, folgt dann aus dem Fundamentallemma.

\section{Distributionen}

Setze $\D(\Omega)\coloneqq C_c^\infty(\Omega)$. Es gelte $\varphi_j \to \varphi$ in $\D(\Omega)$, falls es $K\subseteq \Omega$ kompakt gibt mit $\supp \varphi_j \subseteq K$ für alle $j$ und $D^\alpha \varphi_j \to D^\alpha \varphi$ gleichmäßig für alle Multiindices $\alpha$.

Definiere $\D'(\Omega)\coloneqq \{ T: \D(\Omega) \to \C \text{ stetig, linear} \}$. Die Elemente von $\D'(\Omega)$ heißen \emph{Distributionen}. Beispiele: Ist $a\in \Omega$, so definiert $\langle \delta_a, \varphi \rangle \coloneqq \varphi(a)$ die \emph{Dirac-Delta-Distribution in $a$}, speziell: $\delta \coloneqq \delta_0$. Ist $f\in L^1_{\loc}(\Omega)$, dann ist die \emph{reguläre Distribution zu $f$} definiert durch $\langle T_f, \varphi \rangle \coloneqq \int f\varphi \, \mathrm{dx}$. Es gilt $1/x\not\in L^1_{\loc}(\R)$; Definiere Distribution über Cauchy-Hauptwert: $\langle \pv 1/x, \varphi \rangle \coloneqq \lim_{\varepsilon \to 0} \int_{|x|\geq \varepsilon} \varphi(x)/x \, \mathrm{dx}$.

Der Raum $\D'(\Omega)$ trägt die schwach-*-Topologie. Beispiele: $f_j\to f$ in $L^1_{\loc}$, dann $T_{f_j}\to T_f$ in $\D'(\Omega)$. Ist $\varphi$ Mollifier, so gilt $T_{\varphi_\varepsilon}\to \delta$.

\subsection{Operationen auf $\D'(\Omega)$}

\begin{enumerate}
    \item Ableitung: $\langle D^\alpha T, \varphi \rangle \coloneqq (-1)^{|\alpha|} \langle T, D^\alpha \varphi \rangle$.
    \item Multiplikation mit $f\in C^\infty$: $\langle f\cdot T, \varphi \rangle \coloneqq f(0) \langle T, \varphi \rangle$.
\end{enumerate}

Beispiele zur Ableitung: Ist $|\alpha| \leq k$, $f\in H^k$, dann: $D^\alpha T_f = T_{D^\alpha f}$, für die \emph{Heavyside-Funktion} $H(x)\coloneqq 1$, falls $x>0$, sonst $0$, gilt $H'=\delta$, für die Delta-Distribution gilt wiederum $\langle D^\alpha \delta, \varphi \rangle = (-1)^{|\alpha|} D^\alpha \varphi(0)$.

\subsection{Faltung}

Idee: Für $f\in L^1_{\loc}$, $\varphi\in \D$ ist $(f*\varphi)(x)\coloneqq \int f(y)\varphi(x-y) \, \mathrm{dy}$. Es ist $x-y$ eine Spiegelung und Translation um $x$ von $y$. Definiere Spiegelung und Translation auf Distributionen und verallgemeinere zu $(T*\varphi)(x)\coloneqq \langle T, \tau_x \tilde\varphi \rangle$, wobei $\tilde f(y) \coloneqq f(-y)$ sowie $\tau_x f(y) \coloneqq f(y-x)$. Es gilt $(T*\varphi)\in C^\infty$, $D_j (T*\varphi) = (D_j T)*\varphi = T*(D_j \varphi)$.

\subsubsection{Fundamentallösungen}

Sei $A$ Differentialoperator, $f\in \D(\R^n)$. Ist $T$ Distribution mit $AT=\delta$, denn heißt $T$ \emph{Fundamentallösung} von $A$ und $u\coloneqq T*f \in C^\infty(\R^n)$ löst $Au=f$ distributionell (aber Lösung ist eine reguläre Distribution!). Es gilt der Satz von \emph{Milgrange-Ehrenpreis}: Jeder Differentialoperator mit konstanten Koeffizienten hat eine Fundamentallösung.

\subsubsection{Träger von Distributionen}

Sei $T\in \D'(\Omega)$, definiere $0_T\coloneqq \{ x\in\Omega: \text{ es ex. offene Umgebung } V \text{ von } x \text{ mit } T_V = 0 \}$, wobei $\langle T_V, \varphi \rangle \coloneqq \langle T, \tilde\varphi \rangle$ mit $\varphi\in \D'(V)$, $\tilde\varphi$ Fortsetzung von $\varphi$ auf $\Omega$ durch $0$. Setze $\supp T\coloneqq \Omega \setminus 0_T$, diese Menge heißt \emph{Träger} von $T$. Beispiele: Für $f\in L^1_{\loc}(\Omega)$ gilt $\supp T_f = \supp f$ (zeige die Gleichheit über Kontaposition (also $x\not \in \supp f$ bzw. $\supp T_f$) und nutze Fundamentallemma bzw. $|\int_V f\varphi|\leq \|\varphi\|_\infty \int |f| = 0$). Für $x\in\Omega$ gilt $\supp \delta_x = \{ x \}$. Für $T\in \D'(\Omega)$, $\varphi \in \D$ gilt: Ist $\supp T \cap \supp \varphi = \emptyset$, dann $\langle T, \varphi \rangle = 0$.

Definiere $\E(\Omega) \coloneqq C^\infty(\Omega)$. Es gelte $\varphi_j \to \varphi$ in $\E(\Omega)$, falls für alle $K\subseteq \Omega$ kompakt gilt: $\|D^\alpha \varphi_j - D^\alpha \varphi \|_{C(K)} \to 0$.  Identifiziere $\E'(\Omega)$ mit den Distributionen mit kompaktem Träger (durch Cut-Off Funktion auf dem Träger der Distribution können Elemente aus $\E(\Omega)$ mit solchen aus $\D(\Omega)$ identifiziert werden).

\subsubsection{Faltung von Distributionen mit kompaktem Träger}

Motivation: Sind $f,g\in L^1_\mathrm{loc}(\R^n)$ und $\varphi\in \D$, so gilt $$\langle f\ast g, \varphi \rangle = \int \int f(y) g(x-y) \varphi(x) \mathrm{d}y \mathrm{d}x = \int f(y) \int \tilde g(y-x) \varphi(x) \mathrm{d}x \mathrm{d}y = \langle f, \tilde g \ast \varphi \rangle,$$ definiere daher auch allgemein $\langle T\ast S, \varphi \rangle \coloneqq \langle T, \tilde S \ast \varphi \rangle$, sofern mindestens eine von beiden Distributionen kompakten Träger hat.

Hat immer höchstens eine Distribution keinen kompakten Träger, so gelten die bekannten Regeln (Kommutativität, Assoziativität, Summe der Träger, ...).  Sind die Träger nicht kompakt, so gilt Assoziativität nicht: $1*(\delta'*H)=1*\delta=1\neq 0=0*H=(1*\delta')*H$.

\subsection{Schwartz-Raum}

Sei $\Sc\coloneqq \Sc(\R^n)\coloneqq \{ f\in C^\infty(\R^n): |f|_{\alpha,\beta} \coloneqq \sup |x^\beta D^\alpha f| < \infty \text{ für alle Multiindices } \alpha, \beta \}$ der \emph{Schwartz-Raum}. Definiere $|f|_m \coloneqq \sup_{|\alpha|,|\beta| \leq m} |f|_{\alpha,\beta}$. Es konvergiere $f_j \to f$ in $\Sc$, falls $|f_n-f|_m \to 0$ für alle $m\in \N$. Es gilt offensichtlich $\D \subseteq \Sc$, aber $\D \neq \Sc$, denn $x\mapsto e^{-|x|^2} \in \Sc\setminus \D$.


\subsubsection{Fouriertrafo auf $S$}

Definiere zu $u$ die (anti-symmetrische) Fouriertrafo durch $\hat u(\xi)\coloneqq \int e^{-ix\xi}u(x) \, \mathrm{dx}$ für $\xi\in\R^n$. Die Fouriertrafo ist Isomorphismus von $\Sc$ nach $\Sc$, $(\hat \cdot )^{-1} = \check\cdot$.

Es ist $\Sc$ abgeschlossen unter Faltung und Multiplikation. Es gilt der Faltungssatz $(f*g)^{\mbox{\textasciicircum}} = \hat f \cdot \hat g$ sowie $(f\cdot g)^{\mbox{\textasciicircum}} = (2\pi)^{-n} \, \hat f \cdot \hat g$ sowie die Ableitungsregeln $(D^\alpha u)^\circum(\xi) = i^{|\alpha|} \xi^\alpha \hat u(\xi)$ als auch $(x^\alpha u)^\circum(\xi) = (-i)^{|\alpha|} D^\alpha \hat u(\xi)$. Es gilt die \emph{Parseval/Plancherel} Gleichung: $\int f\bar g \, \mathrm{dx} = (2\pi)^{-n} \int \hat f \bar{\hat g} \, \mathrm{d\xi}$. Außerdem gilt für $f,g\in \Sc$ die Identität $\int \hat f g = \int f \hat g$, sie wird gleich die Motivation für die Definition der Fouriertrafo einer temperierten Distribution sein. 

\subsubsection{Fouriertrafo auf $L^2$}

Approximiere $f\in L^2(\R^n)$ mit Testfunktionen. Aus der Konvergenz in $L^2(\R^n)$ folgt Konvergenz der regulären Distributionen in $\Sc'$, also $\hat f \in L^2(\R^n)$.

\subsection{Temperierte Distributionen}

Betrachte den Dualraum $\Sc'$ von $\Sc$. Die Elemente heißen temperierte Distributionen. Es gilt $\Sc'\neq \D'$, denn $e^x\not\in \Sc'$ (denn $\varphi\coloneqq e^{-x^2/2}\in \Sc$, aber $\langle e^x, \varphi \rangle = \int e^{x^2/x} = \infty$), jedoch $e^x\, e^{ie^x} \in \Sc'$ (ÜA G18, das Problem liegt also nicht im betragsmäßigen Wachstum), also auch nicht $L^1_{\loc} \hookrightarrow \Sc'$. Wir statten $\Sc'$ mit schwach-*-Topologie aus.

Sei $p$ Polynom, $\psi\in \Sc$, dann sind folgende Operationen auf $\Sc'$ definiert:

\begin{itemize}
    \item Multiplikation mit Polynom: $\langle pT, \varphi \rangle \coloneqq \langle T, p\varphi \rangle$.
    \item Multiplikation mit Schwartz-Funktion: $\langle \psi T, \varphi \rangle \coloneqq \langle T, \psi\varphi \rangle$.
    \item Fouriertrafo: $\langle \hat T, \varphi \rangle \coloneqq \langle T, \hat\varphi \rangle$.
    \item Ableitung: $\langle D^\alpha T, \varphi \rangle \coloneqq (-1)^{|\alpha|} \langle T, D^\alpha \varphi \rangle$.
\end{itemize}

Die Fouriertrafo ist Isomorphismus $\Sc'\to\Sc'$. Es gelten folgende Regeln:

\begin{itemize}
    \item $\F(D^\alpha T) = (ix)^\alpha \F(T)$,
    \item $\F(x^\alpha T) = (-i)^{|\alpha|} D^\alpha \F(T)$,
    \item für $T\in \Sc$ gilt: $T_{\hat S} = \widehat T_S$ und
    \item für $R\in \Sc'$ mit kompaktem Träger gilt: $(T*R)\in \Sc'$ und $\F(T*R)=\F(T)\F(R)$.
\end{itemize}

\subsubsection{Beispiele für die Fouriertrafo auf $\Sc'$}

\begin{itemize}
    \item $\hat\delta = 1$, denn $\langle \hat\delta, \varphi \rangle = \hat \varphi(0) = \int e^{-ix0} \varphi(x) \mathrm{d}x = \langle 1, \varphi \rangle$,
    \item $p(x)=\sum_{|\alpha| \leq m} a_\alpha (x^\alpha 1)$ Polynom, dann $\hat p = \sum_{|\alpha| \leq m} a_\alpha i^{|\alpha|} D^\alpha \delta$.
\end{itemize}

\subsubsection{Fundamentallösungen}

Ist $AT=\delta$, dann $p(i\xi)\hat T = 1$, löse algebraische Gleichung. Rücktrafo liefert Fundamentallösung.

\subsection{Einbettung von Funktionenräume in die Distributionenräume}

Es gilt: $\D \subset \Sc \subset L^p \hookrightarrow \Sc' \hookrightarrow D'$.

Die erste Einbettung ist klar. Dass $\Sc \subset L^\infty$ ist auch klar. Außerdem gilt $\Sc \subset L^1$ nach Hölder: Schreibe $f(x)=(1+|x|^2)^{-N}(1+|x|^2)^Nf(x)$, für $N$ groß genug ist der erste Faktor in $L^1$ und da $(1+|x|^2)^N$ Polynom ist der hintere Faktor beschränkt. Mit Interpolation folgt also $\Sc \subset L^p$. Mit Hölder und $\Sc \subset L^q$ folgt $L^p \hookrightarrow \Sc'$. Schließlich ist $\Sc'\to \D'$ wieder klar.

\subsection{Fouriertransformation auf verschiedenen Räumen}

Es gilt $\F: L^1(\R^n) \to C_0(\R^n)$ nach Riemann-Lebesgue mit $\F(f)(x)\coloneqq \int_{\R^n} e^{i\langle x,y \rangle}f(y) \, \mathrm{dy}$. Eingeschränkt auf $L^1(\R^n) \cap L^2(\R^n)$ ist der Wertebereich $L^2$ und die Abbildung ist isometrisch. Es gibt eine Fortsetzung zu einem unitären Operator $\F: L^2(\R^n)\to L^2(\R^n)$. Wegen $L^p$-Interpolation folgt außerdem $\F: L^p(\R^n)\to L^q(\R^n)$ stetig für $1 \leq p \leq 2$ (beachte $C_0(\R^n) \hookrightarrow L^\infty(\R^n)$). Von $\Sc\to \Sc$ ist $\F$ ebenfalls Isomorphismus, somit auch von $\Sc'\to \Sc'$.

Für $f,g\in L^1(\R^n)$ respektive $f,g\in \Sc$ gilt $\int \hat f g = \int f \hat g$ (zum Beweis wendet man Fubini auf die konkrete Darstellung der Fourier-Trafo an).

Die Ableitungsregel gilt auf $\Sc$ immer und auf $H^k$ für Multiindices $\alpha$ mit $|\alpha| \leq k$.

\section{nichtlineare Randwertprobleme}

\subsection{Fixpunktsätze}

Es gilt der Browersche Fixpunktsatz: Ist $B$ die abgeschlossene Einheitskugel in $\R^n$ und $T:B\to B$ stetig, so besitzt $T$ einen Fixpunkt. Man zeigt damit den Fixpunktsatz von Schauder: Sei $X$ B.R. und $K\subseteq X$ nicht-leer, kompakt, konvex sowie $T:K\to K$ stetig, so besitzt $T$ einen Fixpunkt. Es gilt außerdem folgende Variante: Sei $X$ B.R., $T:X\to X$ stetig und kompakt sowie $\{ u\in X: u = \alpha Tu \text{ für ein } \alpha \in [0,1]\}$ beschränkt, dann besitzt $T$ einen Fixpunkt.

\subsubsection{Anwendung auf nichtlineares Problem}

Betrachte $-\laplace u = f(u), u|_\Omega = 0$, wobei $f$ (Lipschitz?) stetig, beschränkt und $\Omega$ beschränkt ist. Beweisidee: Finde $M_0$ mit $C\coloneqq \{ u\in H^1_0(\Omega): \|\nabla u\| \leq M_0\}$ nicht-leer, kompakt und konvex. $T = (-\Delta)^{-1}(f(\cdot)): H^1_0\to H^1_0$ ist stetig, Schauder liefert Fixpunkt.

\subsection{Methode der Unter- und Oberlösungen}

Löse wieder $-\laplace u = f(u)$ in $\Omega$, $u=0$ auf $\partial\Omega$. Wir setzen die Existenz von Unter- und Oberlösungen voraus, um eine Lösung zu ermitteln, die - mehr oder weniger - konstruktiv ist, und zwar durch monotone Limiten.

Eine Funktion $\underline{u}\in H^1(\Omega)$ heißt \emph{schwache Unterlösung}, falls für $v\in H^1_0(\Omega)$ mit $v\geq 0$ fast überall gilt, dass $\int \nabla \underline{u} \nabla v \leq \int f(\underline{u})v$, analog \emph{schwache Oberlösung}.

\subsection{Nichtexistenz glatter Lösungen von nichtlinearen Gleichungen}

Durch Ausnutzung von Haupteigenwert und Haupteigenfunktion von $-\laplace$ lässt sich zeigen, dass $u_t -\laplace u = u^2$ in $(0,T)\times \Omega$, $u=0$ auf $(0,T)\times \partial\Omega$, $u(0,x)=u_0(x)$ in $\Omega$ keine glatte Lösung besitzt, indem man ein Blowup-Argument macht.

\section{Maximumsprinzipien}

\subsection{elliptische Operatoren}

Betrachten \emph{elliptische Operatoren 2. Ordnung}, d.h. Operatoren der Form

$$Au(x)\coloneqq -\sum_{i,j=1}^n a_{i,j}(x)\partial_j \partial_i u(x) + \sum_{i=1}^n b_i(x) \partial_i u(x) +c(x) u(x),$$

wobei wir $a_{i,j}=a_{j,i}$ annehmen.

Der Operator $A$ heißt \emph{gleichmäßig elliptisch}, falls ein $\mu > 0$ existiert, mit

$$\sum_{i,j=1}^n a_{i,j}(x)\xi_i\xi_j \geq \mu |\xi|^2.$$

Es gilt das \emph{schwache Maximumsprinzip}: Ist $\Omega$ beschränktes Gebiet und $A$ gleichmäßig elliptisch mit stetigen Koeffizienten, wobei $c\equiv 0$, dann gilt für $u\in C^2(\Omega) \cap C(\overline{\Omega})$ mit $Au \leq 0$ in $\Omega$, dass das Maximum auf dem Rand von $\Omega$ angenommen wird.

Das \emph{Lemma von Hopf} besagt, dass für ein $u\in C^2(\Omega) \cap C^1(\overline{\Omega})$ mit $Au\leq 0$ in $\Omega$ und $x_0\in \partial\Omega$ mit $u(x_0) > u(x)$ für $x\in \Omega$ gilt, dass $\frac{\partial u}{\partial \nu}(x) > 0$, wobei die innere Kugelbedingung in $x_0$ gelten soll, d.h. es gibt eine offene Kugel $K\subseteq \Omega$ mit $x_0 \in \partial K$.

Aus dem Lemma von Hopf folgt das \emph{starke Maximumsprinzip}: Sind $u,A,\Omega$ wie im schwachen Maximumsprinzip und wird das Maximum in einem inneren Punkt angenommen, so ist $u$ konstant in $\Omega$.

\subsection{parabolische Operatoren}

Betrachte Operator

$$Lu(t,x)\coloneqq \partial_t u(t,x) -\sum_{i,j=1}^n a_{i,j}(t,x)\partial_j \partial_i u(t,x) + \sum_{i=1}^n b_i(t,x) \partial_i u(t,x)$$

auf Gebiet $G\subseteq \R_+ \times \R^n$ mit stetigen Koeffizienten und $a_{i,j}=a_{j,i}$.

Der Operator $L$ heißt \emph{gleichmäßig parabolisch}, falls ein $\mu > 0$ existiert, mit

$$\sum_{i,j=1}^n a_{i,j}(t,x)\xi_i\xi_j \geq \mu |\xi|^2.$$

Es gilt das \emph{schwache Maximumsprinzip}: Ist $u\in C^2(G) \cap C(\overline{G})$ mit $Lu \leq 0$, so nimmt $u$ sein Maximum auf dem parabolischen Rand an.

Außerdem gilt ein starkes Maximumsprinzip von Hopf: Wird in einem inneren Punkt das Maximum angenommen, so auch in jedem Punkt, der durch horizontale und nach oben gerichtete vertikale Segmente verbunden werden kann.

\section{Halbgruppentheorie}

\subsection{Brownsche Bewegung}

Beispiel für Brownsche Bewegung: $P(t,x,y)\coloneqq G_t(x-y)$. Auf $BUC(\R^n)$ definiert $(T(t)f)(x)\coloneqq \int P(t,x,y)f(y) \, \mathrm{dy}$ eine stark stetige Halbgruppe. Für ein $c > 0$ ist $A=c\cdot\triangle$ der Erzeuger dieser Halbgruppe, insbesondere $P(t,x,y) = G_{ct}(x-y)$.

\subsection{abstraktes Cauchy-Problem}

Betrachte \emph{abstraktes Cauchy-Problem}: $X$ Banachraum, $A:D(A) \to X$ Operator, dann soll $u: [0,\infty) \to X$ gefunden werden mit $u(t)\in D(A)$ und $u'(t) = Au(t)$ für $t>0$ sowie $u(0)=u_0$ für ein $u_0\in X$. Beispiel: $X=L^2(\Omega)$, $A=\triangle$, $D(A)=H^2(\Omega)$. Aus einer PDE wird also eine banachraumwertige ODE.

\subsection{$C_0$-Halbgruppen}

Eine Familie $T\coloneqq (T(t))_{t\geq 0}$ von beschränkten, linearen Operatoren auf $X$ heißt \emph{$C_0$-Halbgruppe} auf $X$, falls gilt: $T(0)=\id$, $T(s+t)=T(s)T(t)$ sowie für alle $f\in X$: $t\mapsto T(t)f$ stetig. Man nennt $T$ \emph{Kontraktionshalbgruppe}, falls $\|T(t)\| \leq 1$ für alle $t\geq 0$ gilt.

\subsection{Generator einer Halbgruppe}

Setze $D(A)\coloneqq \{ f\in X: \lim_{t\to 0} \frac{1}{t}(T(t)f-f) \, \text{existiert in } X \}$ und definiere $Af \coloneqq \lim_{t\to 0} \frac{1}{t}(T(t)f-f)$ auf $D(A)$. Dann heißt $(D(A),A)$ der Generator von $T$.

Es gilt $\frac{d}{dt} T(t)f = AT(t)f$, also ist $u(t)\coloneqq T(t)f$ Lösung des abstrakten Cauchy-Problems von $A$ zum Anfangswert $f$.

Der Generator ist dicht definiert und abgeschlossen. Beweisidee: Dicht definiert: Zeige $\int_0^t T(s)f \, \mathrm{ds}$ ist in $D(A)$, dann konvergiert $\frac{1}{t} \int_0^t T(s)f \, \mathrm{ds}$ gegen $f$. Abgeschlossen: Schreibe $T(t)f_n-f_n$ via Hauptsatz und nutze, dass $\frac{d}{dt} T(t)f_n$ die DGL löst.

\subsection{Hille-Yosida-Theorem}

Sei $A$ ein dicht definierter Operator auf $X$. Dann erzeugt $A$ eine $C_0$-Halbgruppe mit $\|T(t)\| \leq 1$ gdw. $(0,\infty) \subseteq \rho(A)$ und $\|(\lambda-A)^{-1}\| \leq \frac{1}{\lambda}$ für alle $\lambda > 0$. 

Beweisidee: Hinrichtung: Zeige $R_\lambda f\coloneqq \int_{0}^{\infty} e^{-\lambda t}T(t)f \, \mathrm{dt}$ ist die Resolvente zu $\lambda$ von $A$ ausgewertet an $f$. Für jene Darstellung folgen die Eigenschaften leicht. Rückrichtung: Regularisiere $A$ durch den beschränkten Operator $A_\lambda \coloneqq \lambda A (\lambda-A)^{-1}$. Es konvergiert $A_\lambda$ stark gegen $A$, $T_\lambda(t) \coloneqq e^{tA_\lambda}$ definiert stark stetige Kontraktionshalbgruppe zu $A_\lambda$. Es konvergiert $T_\lambda$ gegen eine Halbgruppe $T$, dessen Generator $A$ ist.

\subsubsection{Motivation für die Yosida-Approximation und Hille-Ansatz}

In $\C$ gilt $\frac{\lambda}{\lambda-a}=1+\frac{a}{\lambda-A}\to 1$ für $\lambda\to\infty$. Dies motiviert $\lambda R(\lambda,A)f\to f$. Wir wollen $A$ stark approximieren, also schreiben wir $\lambda AR(\lambda,A)$ und da $A$ und $R(\lambda,A)$ auf $D(A)$ kommutieren, folgt $\lambda AR(\lambda,A)\to A$ stark auf $D(A)$. Durch ausmultiplizieren erhalten wir außerdem $\lambda AR(\lambda,A)=\lambda^2R(\lambda,A)-\lambda$, also sind dies beschränkte Operatoren.

Hille ging einen direkteren Weg: Im Reellen gilt die Formel $\e^x=\lim_n (1+x/n)^n=\lim_n (1-x/n)^{-n}$. Für $x$ wollen wir $tA$ einsetzen. Wir schreiben dazu $(1-\frac{t}{n}x)^{-1}=\frac{n}{t}(\frac{n}{t}-x)^{-1}$, stellen fest, dass im Wesentlichen die Resolvente übrig bleibt, und erhalten die Hille-Approximation $\e^{tA}=\lim_n (\frac{n}{t}R(\frac{n}{t},A))^n$.

\subsubsection{Motivation: Lagrange-Ansatz zur Lösung der Transportgleichung}

Betrachte $\partial_t u(t,x)-\partial_x u(t,x)=0$. Wir wollen eine Lösung via $u(t,x)=\e^{t\dd}f$ finden. Dazu machen wir eine formale Potenzreihenrechnung: $$\e^{t\partial_x}f(x)=\sum_{n=0}^\infty \frac{t^n\partial_x^n}{n!}f(x)=\sum_{n=0}^\infty \frac{f^{(n)}(x)}{n!}t^n=f(x+t).$$ Die letzte Gleichheit ist dabei Taylor. Diese Formel definiert die Linkstranslationshalbgruppe und ist in der Tat die Lösung obiger Gleichung.

\subsection{Anwendung Hille-Yosida auf parabolisches Problem}

Löse das abstrakte Cauchy-Problem durch Anwendung von Hille-Yosida. Resolventenbedingung wird auf Lösung der Resolventengleichung reduziert, für die es schwache Lösungstheorie gibt. Normabschätzung folgt aus der Ungleichung $a(u,u) \geq \alpha \|u\|_{H^1_0}^2 - \gamma \|u\|_{L^2}^2$. Nach Hille-Yosida gibt es also zu $A$ eine stark-stetige Halbgruppe, die Lösung des abstrakten Cauchy-Problems ist.


\end{document}
